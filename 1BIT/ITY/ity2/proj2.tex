\documentclass[a4paper,11pt,twocolumn]{article}

\usepackage[left=1.5cm,text={18cm, 25cm},top=2.5cm]{geometry}

\usepackage[utf8]{inputenc}
\usepackage[czech]{babel}
\usepackage[IL2]{fontenc}
\usepackage{times}

\usepackage{dtklogos,amsmath,amsthm,mdwlist,enumitem,amssymb}
\pagenumbering{arabic}

\theoremstyle{definition}
\newtheorem{definice}{Definice}[section]

\theoremstyle{plain}
\newtheorem{veta}{Věta}

\theoremstyle{definition}
\newtheorem{algoritmus}[definice]{Algoritmus}

\begin{document}
\begin{titlepage}

\begin{center}
\textsc{\Huge{Fakulta informačních technologií}}\\
\vspace{\stretch{0.0055}}
\textsc{\Huge{Vysoké učení technické v~Brně}}\\
\vspace{\stretch{0.382}}
\LARGE{Typografie a publikování\,--\,2. projekt}\\
\LARGE{Sazba dokumentů a~matematických výrazů}
\vspace{\stretch{0.618}}
\end{center}

{\Large 2015 \hfill David Kolečkář}
\vspace{\stretch{0.02}}

\end{titlepage}

\section*{Úvod}
V této úloze si vyzkoušíme sazbu titulní strany, matematických vzorců, prostředí a dalších textových struktur obvyklých pro technicky zaměřené texty (například rovnice (\ref{rovnice1}) nebo definice \ref{definice1_1} na straně \pageref{definice1_1}).\par
\par
Na titulní straně je využito sázení nadpisu podle optického středu s využitím zlatého řezu. Tento postup byl probírán na přednášce.
\par

\section{Matematický text}
Nejprve se podíváme na sázení matematických symbolů a výrazů v plynulém textu. 
Pro množinu $V$ označuje $\mbox{card}(V)$ kardinalitu $V$.
Pro množinu $V$ reprezentuje $V^\ast$ volný monoid generovaný množinou $V$ s~operací konkatenace.
Prvek identity ve volném monoidu $V^\ast$ značíme symbolem $\varepsilon$.
Nechť $V^+ = V^* - \{\varepsilon\}$. Algebraicky je tedy $V^+$ volná pologrupa generovaná množinou $V$ s~operací konkatenace.
Konečnou neprázdnou množinu $V$ nazvěme \emph{abeceda}.
Pro $\mathit{w} \in V^*$ označuje $|w|$ délku řetězce $w$. Pro $W \subseteq V$ označuje $\mbox{occur}(w, W)$ počet výskytů symbolů z~$W$ v~řetězci $w$ a $\mbox{sym}(w, i)$ určuje $i$-tý symbol řetězce $w$; například $\mbox{sym}(abcd, 3) = c$.
\par
Nyní zkusíme sazbu definic a vět s využitím balíku \texttt{amsthm}.

\begin{definice}\label{definice1_1}
\emph{Bezkontextová gramatika} je čtveřice $G = (V,T,P,S)$, kde $V$ je totální abeceda,
$T \subseteq V$ je abeceda terminálů, $S \in (V - T)$ je startující symbol a
$P$ je konečná množina \emph{pravidel} tvaru $q\colon A \rightarrow \alpha$, kde $A \in (V - T)$, $\alpha \in V^*$ a $q$ je návěští tohoto pravidla.
 Nechť $N = V - T$ značí abecedu neterminálů. Pokud $q\colon A \rightarrow \alpha \in P$, $\gamma$,\,$\delta \in V^*$, $G$ provádí derivační krok z~$\gamma{A}\delta$ do $\gamma\alpha\delta$ podle pravidla $q\colon A \rightarrow \alpha$, symbolicky píšeme $\gamma{A}\delta \Rightarrow \gamma\alpha\delta\,\,[q\colon A \rightarrow \alpha]$ nebo zjednodušeně $\gamma{A}\delta \Rightarrow \gamma\alpha\delta$.
Standardním způsobem definujeme $\Rightarrow^m$, kde $m \geq 0$. Dále definujeme tranzitivní uzávěr $\Rightarrow^+$ a~tranzitivně-reflexivní uzávěr $\Rightarrow^*$. 

\end{definice}
\par
Algoritmus můžeme uvádět podobně jako definice textově, nebo využít pseudokódu vysázeného ve vhodném prostředí (například  \texttt{algorithm2e}).
\par

\begin{algoritmus}
\emph{Algoritmus pro ověření bezkontextovosti gramatiky. Mějme gramatiku $G = (N, T, P, S)$.}
\begin{enumerate}[font=\itshape] %package enumitem
  \item \label{krok1}\emph{Pro každé pravidlo $p \in P$ proveď test, zda $p$ na levé straně obsahuje právě jeden symbol z~$N$.}
  \item \emph{Pokud všechna pravidla splňují podmínku z~kro\-ku \ref{krok1}, tak je gramatika $G$ bezkontextová.}
\end{enumerate}
\end{algoritmus}
\par
\begin{definice}
  \emph{Jazyk} definovaný gramatikou $G$ definujeme jako $L(G) = \{w \in T^*\,\,|\,\,S \Rightarrow^* w\}$.
\end{definice}
\par

\subsection{Podsekce obsahující větu}
\begin{definice}
Nechť $L$ je libovolný jazyk. $L$ je \emph{ bezkontextový jazyk},když a jen když $L = L(G)$, kde $G$ je libovolná bezkontextová gramatika.
\end{definice}
\par
\begin{definice}
Množinu $\mathcal{L}_{CF} = \{L|L$ je bezkontextový jazyk$\}$ nazýváme \emph{třídou bezkontextových jazyků}.
\end{definice}
\par
\begin{veta}\label{veta1}
\emph{Nechť} $L_{abc} = \{a^nb^nc^n|n \geq 0\}$. \emph{Platí, že} $L_{abc} \not\in \mathcal{L}_{CF}$.
\end{veta}
\par
\begin{proof}
 Důkaz se provede pomocí Pumping lemma pro bezkontextové jazyky, kdy ukážeme, že není možné, aby platilo, což implikuje pravdivost věty \ref{veta1}.
\end{proof}
\par

\section{Rovnice a odkazy}
Složitější matematické formulace sázíme mimo plynulý text. Lze umístit několik výrazů na jeden řádek, ale pak je třeba tyto vhodně oddělit, například příkazem \verb|\quad|.
\par
$$\sqrt[x^2]{{y}{^3_0}}\quad\mathbb{N} = \{0, 1, 2,\ldots\} \quad x^{y^y} \neq x^{yy} \quad z_{i_j} \not\equiv z_{ij}$$
\par
V~rovnici (\ref{rovnice1}) jsou využity tři typy závorek s~různou explicitně definovanou velikostí.
\par
\begin{eqnarray}
\bigg\{\Big[\big(a + b\big) * c\Big]^d + 1 \bigg\}& = & x\label{rovnice1} \\
\lim_{x \to \infty}\frac{\sin^2 x + cos^2 x}{4} &=& y \nonumber
\end{eqnarray}
\par
V~této větě vidíme, jak vypadá implicitní vysázení limity $\lim_{n \to \infty}{f(n)}$ v~normálním odstavci textu. Podobně je to i s~dalšími symboly jako $\sum_1^n$ či $\bigcup_{A\in{B}}$.
V~případě vzorce $\lim\limits_{x \to 0}\frac{\sin x}{x}=1$ jsme si vynutili méně úspornou sazbu příkazem \verb|\limits|.
\par
\begin{eqnarray}
  \int_a^b f(x)\,\mathrm{d}x &=& -\int\limits_b^a f(x)\,\mathrm{d}x \\
  \left(\sqrt[5]{x^4}\right)' = \left(x^{\frac{4}{5}}\right)' &=& \frac{4}{5}x^{-\frac{1}{5}} = \frac{4}{5\sqrt[5]{x}}\\
 \overline{\overline{A \vee B}} &=& \overline{\overline{A} \wedge \overline{B}}
\end{eqnarray}
\par


\section{Matice}
Pro sázení matic se velmi často používá prostředí \texttt{array} a závorky (\verb|\left|, \verb|\right|). 
\par
$$\left(
  \begin{array}{ccc}
    a + b & b - a \\
    \widehat{\xi + \omega} & \pi \\
    \vec{a} & \overleftrightarrow{AC}\\
    0 & \beta
    \end{array}
\right)$$

$$\mathbf{A}=\left\|\begin{array}{cccc}
a_{11} & a_{12} & \ldots & a_{1n} \\
a_{21} & a_{22} & \ldots & a_{2n} \\
\vdots & \vdots & \ddots & \vdots \\
a_{m1} & a_{m2} & \ldots & a_{mn}
\end{array}\right\|$$

$$\left|\begin{array}{cc}
t & u \\ 
v & w
\end{array}\right| = tw - uv$$

Prostředí \texttt{array} lze úspěšně využít i jinde.
\par
$$\binom{n}{k} = \left\{ 
\begin{array}{l l}
  \displaystyle\frac{n!}{k!(n-k)!} & \quad \mbox{pro $0 \leq k \leq n$}\\
  0 & \quad \mbox{pro $k < 0$ nebo $k > n$}\\
\end{array} \right. $$
\par
\section{Závěrem}
V~případě, že budete potřebovat vyjádřit matema\-tickou konstrukci nebo symbol a nebude se Vám dařit jej nalézt v~samotném \LaTeX{u}, doporučuji prostudovat možnosti balíku maker \AmSLaTeX.
Analogická poučka platí obecně pro jakoukoli matematickou konstrukci v~\TeX{u}.
\end{document} 