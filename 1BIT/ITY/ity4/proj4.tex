\documentclass[a4paper,11pt]{article}

\usepackage[left=2cm,text={17cm, 24cm},top=3cm]{geometry}
\usepackage[utf8]{inputenc}
\usepackage[czech]{babel}
\usepackage[IL2]{fontenc}

\usepackage{times}

\usepackage{natbib}
\bibliographystyle{czplain}
\pagenumbering{arabic}

\begin{document}

\begin{titlepage}
\begin{center}
\fontsize{27}{20}\selectfont\textsc{{Vysoké učení technické v~Brně}}\\
\vspace{\stretch{0.0075}}
\fontsize{22.7}{0}\textsc{\selectfont{Fakulta informačních technologií}}\\
\vspace{\stretch{0.381}}
\LARGE{Typografie a publikování\,--\,4. projekt}\\
\Huge{Bibliografické citace}
\vspace{\stretch{0.616}}
\end{center}
{\Large \today \hfill David Kolečkář}
\end{titlepage}

\section{Typografie}
Typografie byla dříve označení pro tiskárenský průmysl, dnes jde především o nauku písma. Na internetu existuje mnoho článků, které se věnují písmové osnově \cite{Program:pismo}, kompozici textu \cite{Typomil:kompozice} nebo nejčastějším chybám v české typografii \cite{Typ:chyby}  atd. Odborný časopis Svět tisku \cite{Tisk}, je měsíčník o tisku a typografii. K přečtení bych doporučil dva odborné články. Typografie v designu \cite{Clanek1} nebo o typografii webových stránek \cite{Clanek2}.

Protože typografie není jednoduchou záležitostí, vznikla řada programů, které mají za úkol usnadnit autorům práci a pomoci ve správné sazbě dokumentu. 
Jedním z nich je i LaTeX. Ačkoliv uživatele může zpočátku zaskočit složitost systému, oproti klasickému textovému editoru Microsoft Word, jedná se vskutku o výjimečný nástroj. 
Začátečníkům může pomoci česká publikace Latex pro začátečníky \cite{Rybicka}. Nebo populární cizojazyčná kniha \cite{Guide}, která přináší podrobný, názorný a srozumitelný výklad. 
Za zmínku stojí také práce, věnující se problematice přípravy kvalitního českého písma  \cite{Cerny}.
Mezi další nástroje patří například profesionální program Adobe InDesign, který umožňuje automatickou kontrolu typografických pravidel  \cite{Kubova}. 
\newpage

\renewcommand{\refname}{Literatura}
\bibliography{proj4}

\end{document} 